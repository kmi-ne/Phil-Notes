\documentclass[b5j]{ltjsarticle}

\setsansfont{nimbussans}
\DeclareEmphSequence{\sffamily}

\usepackage{xcolor}
\definecolor{myurlcolor}{HTML}{ad0d00}
\definecolor{mylinkcolor}{HTML}{2a4f99}

\usepackage{hyperref}
\hypersetup{
  bookmarksnumbered=true,
  colorlinks=true,
  linkcolor=mylinkcolor,
  urlcolor=myurlcolor
}

\title{\sffamily 読書会について}
\author{田中\ 英矩\thanks{\url{enescu.norimuji@gmail.com}}}
\date{最終更新日:\today}

\begin{document}

\maketitle

\begin{abstract}
  9月3日(水)より,Gallagher \& Zahavi, \textit{The Phenomenological Mind}の読書会を行います.
\end{abstract}

\section{目的}

\begin{itemize}
  \item 心の哲学・現象学・認知科学の主要概念を学習する.
  \item 心や意識の研究における既存の哲学的・科学的アプローチを知り,それらに対する応答として提起される方法論を理解し,検討する.
  \item 現象学と,心の分析哲学・認知科学との間の対話の現在を知る.
\end{itemize}


\section{概要}

\begin{description}
  \item[使用テキスト] Gallagher, S., \& Zahavi, D. (2021). \textit{The Phenomenological Mind} (3rd ed.). Routledge.\par
  非推奨ですが,第1版の邦訳(絶版)があります(石原孝二・宮原克典・池田喬・朴嵩哲\ 訳(2011)『現象学的な心\ 心の哲学と認知科学入門』勁草書房).
  \item[参加条件] 特になし.必ずしも美術部員のみに限る必要はないと思っています.
  \item[初回開催日時] 9/3(水)10:30
  \item[頻度] 基本的に週1回.できるだけ速いペースで回したいです.
  \item[形式] 対面(希望に応じてオンラインを併用)
  \item[場所] 大阪大学美術部部室
  \item[持ち物] テキスト
  \item[参加費] 無料(テキスト費用は各自負担.ただし\nameref{sec_note}も参照)
\end{description}


\section{進行}

輪読+発表型のゼミだと思ってもらえれば大体合ってます.

各回は,発表(前半)とディスカッション(後半)からなります.おおよそ以下の手順で進みます.

\begin{enumerate}
  \item 読書会の趣旨説明(初回,または新規参加者が来た場合のみ).
  \item 発表者が,作成したレジュメをもとに発表を行う(初回は,田中と⿑藤が第1章の発表を行います).
  \item 発表後,オーディエンスによるレビューや質疑応答を行う.
  \item 発表をもとに,全員によるディスカッションで,テキストや発表内容の検討を行う.
  \item 最後に次回発表者を決め,次回までに読んでおく範囲を決める.
\end{enumerate}

次回発表者は,次回までに次のことを行います.

\begin{itemize}
  \item 次回までに読んでおく範囲を読む.
  \item レジュメを作成する.レジュメでは少なくとも,テキストの該当範囲の情報をまとめます.もちろん,まとめに加えて,疑問点,追加情報,意見,自身の問題意識との関わりなども示せるととても良いです.
\end{itemize}
ただし,\emph{余力が許せば,どんどん先の内容も読んで発表していただいても大丈夫です}\footnote{
  想定しているペース(週1・毎回1章ずつ)で順調に進んだとしても,終了までに2か月半かかります.
}.

できれば毎回少なくとも1章ずつ進みたいのですが,1人が1章丸ごと担当するのは重いので,発表者を複数人指名し,担当範囲を分散することが多くなるでしょう.

強制ではありませんが,発表者以外であっても,なるべく多くの参加者がテキストの該当範囲を読んでおくことが望ましいです(が,無理強いはしませんので時間と相談してください).

\section{備考}
\label{sec_note}

\begin{itemize}
  \item テキストは購入することを推奨しますが,入手・購入に難がある方は田中までお声がけください.(ヒント:私は電子版を持っています.後は察してください.)
  \item ある程度固定的なメンバーがいた方がよいですが,一部の回だけ参加する人がいても良いと思います.
  \item ある回に参加できなかった人にも,内容が共有できるようにしたいです.
\end{itemize}


\section{}

\begin{description}
  \item[第1章] 20世紀以降,心に関する研究の方法論がどのように発展してきたかを概観します.また,現象学とはどのような方法論であるかを概説します.
  \item[第2章] 科学者が心や「経験」を研究する際に用いる方法論,特に「現象学」というアプローチが実験の現場でどのように使われるのかを具体的に見ていきます.
  \item[第3章] 「意識」と「自己意識」とは何かというテーマを扱います.現代の哲学で繰り広げられている意識に関する論争を追いながら,この問題への新しいアプローチを探ります.
  \item[第4章] 私たちの意識や経験が,過去・現在・未来をどのように内包しているのかという「時間性」の問題を探ります.ウィリアム・ジェイムズが提唱した概念を手がかりに,現象学の視点から議論を深めます.
  \item[第5章] 私たちの意識が常に対象に向けられているという「志向性」の概念を学びます.これが知覚や記憶といった心の働きにおいてどのように機能し,現代の心身問題の議論にどう繋がるかを見ていきます.
  \item[第6章] 「知覚」を深く掘り下げます.心と身体の繋がりを重視する新しい知覚理論を比較検討し,知覚と想像の違いについても議論します.
  \item[第7章] 私たちが経験する「生きられた身体」と,客観的に見られる身体との違いから「身体性」の問題を考えます.身体のあり方が私たちの認知や経験をどのように形作るのかを探ります.
  \item[第8章] 人間の「行為」をテーマとします.自分が「行為している」という感覚と,自分の身体が「動いている」という感覚の違いを手がかりに,人間の行為の本質に迫ります.
  \item[第9章] 私たちがどのように他者の心を理解するのかを探ります.既存の理論を検討しつつ,現象学に基づいた新しいアプローチを学び,社会的な関わりについても考えます.
  \item[第10章] 哲学から神経科学まで広く議論される「自己」とは何かという問いに挑みます.あらゆる経験の根底にある基本的な自己感覚に注目し,それがどのようにして物語的な自己へと発展していくのかを探ります.
  \item[第11章] これまでの総括を行います.
\end{description}


\end{document}
