\documentclass[b5j, 9pt]{ltjsarticle}

\usepackage{../settings}
\RenewDocumentCommand{\where}{m}{}

\回数{1}

\発表日{2025年9月3日}

\範囲{Chapter 1, pp. 2--6. ``AN OVERSIMPLIFIED SUMMARY OF THE LAST 120 YEARS''}

\begin{document}

\ヘッダー

\begin{abstract}
  19世紀末には心をめぐる哲学・心理学の議論が活発に行われたが,20世紀に入ると分析哲学と現象学が分裂.両者間の交流はほとんどなくなり,敵対すらした.\par
  心理学では,内観への注目の後に,行動主義が支配的になったが,後に認知主義が台頭したことで意識への関心が再燃した(というのが「通説」だが,実際はそう単純ではない).\par
  そして近年,「現象的意識への関心」「身体化された認知」「神経科学の進歩」という3つの要因により,現象学の重要性が再認識され,認知科学との対話が再び活発化している.
\end{abstract}

\tableofcontents


\section{要約}

\subsection{テキストについて}

\begin{description}
  \item[概要] \term{心}についての哲学的問題を探求する\where{¶1}.
  \item[どのように?] 哲学的アプローチだけでなく,科学的知見も積極的に参照する\where{¶1}.また,議論する問題に対して,\term{現象学}の視点を採用する\where{¶2:L5}.
  \item[目的] 現象学と分析的アプローチとの対話を再開させること.
\end{description}


\subsection{分析哲学と現象学の断絶}

\begin{description}
  \item[19世紀末] 心をめぐり,思想家たちは互いに交流し影響を与え合っていた\where{¶1}.\par
  {\small ジェームズ,シュトゥンプ,フッサール,ブレンターノ,フレーゲ,ラッセル,…}
  \item[20世紀に突入] 各思想家のアプローチが分化.特に,
  \begin{itemize}
    \item フッサール:\term{現象学}を創始
    \item フレーゲやラッセル:\term{分析哲学}へと発展
  \end{itemize}
  \item[それ以降] 心の分析哲学と現象学は,互いを無視・敵視する関係に\where{¶2}.
  \begin{itemize}
    \item Marion (1998):現象学が20世紀の哲学をけん引してきたと主張
    \item Smart (1975):現象学は「全くのナンセンス」
    \item Searle (1975):現象学には深刻な限界があり,「破産状態にあるとすら言いたい」
  \end{itemize}
\end{description}


\subsection{心理学の展開の通説と,その批判的検討}

\subsubsection{心理学の展開の通説}

「内観 → 行動主義 → 認知革命」という展開が通説に.

\begin{description}
  \item[内観] 19世紀末--20世紀初頭,実験心理学者は,心についての測定可能なデータを得るのに\term{内観}(introspection)に頼った.
  \item[行動主義] 1913年頃\footnote{Watson (1913)を念頭に置いてのことと思われる[発表者].},\term{行動主義}(behaviorism)が登場し,研究対象を観察可能な行動に限定.\par
  ジョン・ワトソンが主導.50年頃を最盛期に,70年代まで台頭.cf. Watson (1924)
  \item[認知革命] その後(1950年代),\term{認知}的アプローチが,行動主義に取って代わる.\par
  計算モデル・脳科学の進展を背景とし,内的プロセスへの関心が再燃.\par
  {\small → 1980後半--90年代,\term{意識の神経相関}(neural correlates)の特定を目指した.}
\end{description}


\subsubsection{通説への批判}

著者は,この「通説」は「歪曲され,過度に単純化されている」と批判.\footnote{
  \begin{itemize}
    \item 客観的測定は,19世紀の初期の心理学研究でも一般的だった.
    \item 内観は,「内観主義者」自身によってもしばしば問題視された(Wundt (1900)など).
    \item 心の計算論的理解は18世紀まで遡れる.そもそも,意識はそれ以前から関心の的だった.
    \item 「初期の心理学は内観主義的」は,行動主義を推進したかったワトソンの捏造であるという指摘(cf. Costall (2004, 2006)).
    \item 認知主義は,実際には行動主義の継続であるという指摘(Costall (2004)).
    \item 認知科学や中期分析哲学は,行動主義的思考の影響を受けていた.
  \end{itemize}
}

なお,\emph{現象学は内観主義的という理解は,誤解である}(cf. Ch. 2).

\subsection{なぜ現象学は周縁化されてきたか}

\begin{itemize}
  \item 科学と分析哲学者は\term{自然主義}(naturalism)を,現象学者は非/反自然主義を採用する傾向.\par
  → 認知科学の登場時には,心の分析哲学のほうが相性が良かった(特に,計算モデルとの相性).
  \item 分析哲学側も,重要な理論的基盤や概念分析を,認知科学に提供した(例:機能主義).
\end{itemize}
→ 現象学は,この認知科学の枠組みから周縁化され,無関係とみなされた\footnote{
  例外:H. Dreyfus (1967, 1972, 1992)は,AI・認知科学の問題に対する現象学の関連を主張し続けた.
}.


\subsection{最近の現象学の再評価}

最近,以下の3つの要因により,この状況が変化している.

\begin{description}
  \item[現象的意識への関心の再燃] Nagel (1974)を皮切りに,多くの心理学者・哲学者\footnote{
    cf. Marcel \& Bisiach (1988),Dennett (1991),Flanagan (1992),Searle (1992),Strawson (1994),Chalmers (1995)
  }が意識の問題に再び着手.\par
  「経験的側面を科学的に研究する方法は?」→ 現象学的アプローチが重要では\footnote{
    cf. Gallagher (1997),Varela (1996)
  }
  \item[身体化された認知] 90年代に\term{身体化された認知}(embodied cognition)の概念が強まる\footnote{
    cf. Varela et al. (1991),Damasio (1994),Clark (1997)
  }.\par
  メルロ=ポンティに立ち返ることで身体化の重要性を強調.\par
  \item[神経科学の進歩] \term{脳画像化技術}(fMRI、PET)により,被験者の経験報告に依存する様々な実験が可能に.\par
  → 実験構築や,結果の解釈などのため,被験者の経験を知りたいときがある.\par
  → 意識経験を記述するための信頼できる方法:現象学?
\end{description}

\end{document}
